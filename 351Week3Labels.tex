\documentclass{article}

\usepackage{amssymb, amsmath, amsthm, makeidx}

\makeindex

\newtheorem{theorem}{Theorem}
\newtheorem{lemma}[theorem]{Lemma}

\title{Automatic labels}
\date{}

\begin{document}

\maketitle

\begin{abstract}
  This is an example of a abstract.  \index{abstract} An abstract is a short description
  of the contents of the document.  As a general rule, abstracts should be no more than a
  few sentences for papers and a few paragraphs for books.  This document describes how to
  use \LaTeX{'s} self numbering and labeling features.
\end{abstract}

\tableofcontents

\section{Introduction}

This document is designed to illustrate \LaTeX{'s} ability to number and reference
sections, subsections, theorems, equations, and so on.  We will also illustrate
appendices, bibliographies, indices, and footnotes.\footnote{Yes, even footnotes!  Oh,
  lucky day!}  Footnotes \index{footnotes} are created using \verb~\footnote{text}~ and,
if appearing at the end of a sentence, should appear after the period.

\subsection{Compile more than once}

Auxiliary files that store information about labels and page numbers are created the first
time the \LaTeX{} source is compiled.  The next time \LaTeX{} is compiled, those files are
referenced to create the table of contents, section numbers, and theorem numbers.  Because
of this, \LaTeX{} must be compiled two or three times in order to properly display the
labels.  The output shown by the compiler may warn you if it appears that \LaTeX{} should
be run again.

Sections \index{sections} are created with \verb~\section[TOC name]{Section Name}~, where
the optional \verb~[TOC name]~ string changes the name of the section in the table of
contents.  There is also a similar \index{subsections} \verb~\subsection~ command.  To
print a section without a number (which will also not show up in the table of contents),
use \verb~\section*{Section Name}~.  The table of contents is created with the
\verb~\tableofcontents~ command.

\subsection[BibTeX]{Bibliographies using BibTeX}

Bibliographies can created with BibTeX.  \label{bibliographies} \index{BibTeX} To do this,
follow these instructions (which have been auto-magically numbered using \index{enumerate}
the \verb~enumerate~ environment):
\begin{enumerate}
\item create a new file called \verb~file.bib~.
\item In that new file, create or copy and paste bibliography entries in the
  BibTeX format.  Examples of the format can be found in the sample .bib file
  on our web site.

  BibTeX formatted citations can be copy and pasted from paper indexing
  services such as MathSciNet or Google Scholar.
\item Cite authors or works using \verb~\cite{name1, name2}~.  For example,
  some of Euler's \index{Euler} works include
  \cite{euler1741solutio,euler1759force}.  Later, Euler published
  \cite{euler1775meditationes}.  To include in the bibliography without citing,
  use \verb~\nocite~.
\item Place these two lines in the \LaTeX{} file (usually towards the end of
  the file) before the \verb~\end{document}~ command:
\begin{verbatim}
\bibliography{file}
\bibliographystyle{alpha}
\end{verbatim}
Other choices for the bibliography styles are:
\verb~abbrv~, \verb~acm~, \verb~alpha~, \verb~apalike~, \verb~ieeetr~, \verb~plain~, and
\verb~siam~.
\item Compile the \LaTeX{} file, compile the bibliography with BibTeX by either
  \begin{enumerate}
  \item entering \texttt{bibtex file} at the command line (the \verb~.bib~
    extension is not included after \verb~file~), or
  \item finding the bibtex command in a drop down menu (look for a command such
    as \verb~pdflatex+bibtex~),
  \end{enumerate}
  and then compile the \LaTeX{} file two more times (unless this has been done
  automatically by the software).
\end{enumerate}

\subsection{Creating an index}

To create an \index{index} index, place these \index{two} two lines before the
\verb!\begin{document}! command:
\begin{verbatim}
\usepackage{makeidx}
\makeindex
\end{verbatim}

  Mark words for inclusion in an index using the \verb~\index{word}~ command.  The page
  reference will appear on the same page that the \verb~\index{word}~ command is located.
  There are variations on the \verb~\index~ command which can be used to change how the
  entries appear in the index; such options are found on page 87 of the text.

Similar to BibTeX, the index must be compiled with its own command.  To compile
an index, compile the \LaTeX{} file, enter \texttt{makeindex file} at the
command line or use a drop down menu command, then compile the \LaTeX{} file
two more times if necessary.

\section{Referencing other parts of the text}

To mark a location in the text for future reference, insert
\verb~\label{name}~.  The names for each of the labels in the document must be
unique.  To reference this label, use the command \verb~\ref{name}~ to
reference the section (or subsection) and use the \verb~\pageref{name}~ to
reference the page.  For example, we showed how to include bibliographies in
Section \ref{bibliographies} on page \pageref{bibliographies}.

\subsection{Referencing in math mode}

Cross-referencing is \index{cross-referencing} enhanced by the standard
\verb~amssymb~, \verb~amsmath~, and \verb~amsthm~ packages.  To create a
numbered displayed equation in mathematics mode, use
\verb~\begin{equation}..\end{equation}~ instead of the usual \verb`\[..\]`.
The \verb~\label{name}~ command must appear between the \verb~\begin{equation}~
  and the \verb~\end{equation}~ commands.  When later referring to an equation,
the command \verb~\eqref{name}~ produces parentheses around the equation
number.

For example, the following identity can be proved using polar coordinates:
\begin{equation}
\label{Gauss}
\int_{-\infty}^\infty e^{-x^2} \, dx = \sqrt{\pi}.
\end{equation}
The integral in \eqref{Gauss} is known as the Gaussian integral.

Differently named \verb~\label~ commands can be placed in each line in either
the \verb~align~ or \verb~multline~ environments.  To suppress a line number
when using these environments, place the \verb~\nonumber~ command somewhere on
that line.  Instead of a line number, a word or symbol can be placed in
parenthesis on the side of an \verb~align~ or \verb~multline~ command using
\verb~\tag{word}~.\footnote{Use this sparingly!}

For an example of this, we have
\begin{align}
  \left( \int_{-\infty}^\infty e^{-x^2} \, dx \right)^2
  & = \iint_{\mathbb{R}^2} e^{-(x^2+y^2)} \, d(x,y) \tag{$\star$} \label{over the plane} \\
  & = \int_0^{2 \pi} \int_0^\infty e^{-r^2} r \, dr \, d\theta \label{polar} \\
  & = 2 \pi \int_{-\infty}^0 \frac{1}{2} e^u \, du \tag{$u = -r^2$} \\
  & = \pi, \nonumber
\end{align}
where we integrated over the plane $\mathbb{R}^2$ in \eqref{over the plane} and switched
into polar coordinates in \eqref{polar}.

Whenever you have a equation number in the document, you are telling the reader
``remember where this line is!'', and so it is polite to suppress equation
numbers unless those equations are actually referenced in the document.

\subsection{Theorems and proofs}

Theorems, lemmas, exercises, and the like can be created using these steps.
\begin{enumerate}
\item Place \verb~\newtheorem{name1}{Name2}~ before the
  \verb~\begin{document}~.  The first name, \verb~name1~, is the environment
    name and will only used internally in the \LaTeX{} document.  The section
    name, \verb~name2~, is what will be shown to the reader along with a
    number.
  \item  Create a theorem-like statement using the syntax
\begin{center}
  \verb~\begin{name1}[optional] ..text.. \end{name1}~.
\end{center}
\end{enumerate}
For example, we display Theorem \ref{E} next.
\begin{theorem}[Euler]
  \label{E}
If $t$ is a real number, then $e^{i t} = \cos t + i \sin t$.  \index{Cosine}
\index{Sine}
\end{theorem}

Those wanting more control over the appearance of the theorem environment are
referred to page 72 in the text.

The proof of this theorem can be written between \verb~\begin{proof}~ and
  \verb~\end{proof}~ statements.  This will automatically generate an end of
proof symbol.  The placement of the end of proof symbol can be manipulated
using \verb~\qedhere~.\footnote{See page 72 of the text.}  Here is an example
of the proof environment, which serves as a proof of Euler's \index{Euler}
Theorem, our Theorem \ref{E}.

\begin{proof}
  Using the power series \index{power series} for $e^{i t}$, \index{Exponential
    function}
\begin{align*}
  e^{i t} & = 1 + \frac{(i t)^1}{1!} + \frac{(i t)^2}{2!}
  + \frac{(i t)^3}{3!} + \frac{(i t)^4}{4!} + \cdots \\
  & = 1 + i \frac{t^1}{1!} - \frac{t^2}{2!} - i \frac{t^3}{3!} + \frac{t^4}{4!} + \cdots \\
  & = \left( 1 - \frac{t^2}{2!} + \frac{t^4}{4!} - \cdots \right)
  + i \left( t - \frac{t^3}{3!} + \frac{t^5}{5!} - \cdots \right),
\end{align*}
which, using the power series for $\cos t$ and $\sin t$, is equal to
$\cos t + i \sin t$.
\end{proof}

\appendix

\section{Appendix}

All appendix \index{appendix} material must appear between a \verb~\appendix~
command but before the \verb~\end{document}~ command.  After that, include any
appendix sections with the usual \verb~\section{Section Name}~ command.

\nocite{SelfPromotion}

\bibliography{351Week3Labels}
\bibliographystyle{alpha}

\printindex

\end{document}
