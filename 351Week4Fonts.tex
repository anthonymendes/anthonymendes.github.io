% Compile this document with XeLaTeX.
\documentclass[11pt]{article}

\usepackage{amssymb, amsmath, amsthm, mathpazo, mathspec}
\usepackage[left=35mm,top=2cm,right=35mm,bottom=2cm]{geometry}

\theoremstyle{definition}
\newtheorem{problem}{}
\newtheorem{definition}{Definition}

% You may also call fonts that are installed on your machine without needing to
% have the font file in the compile folder
\setmainfont{Lato-Light.ttf}
\setmathsfont(Digits,Latin,Greek)[Numbers={Lining,Proportional}]{Lato-Light.ttf}
\setsansfont{351Week4LDFComicSans.ttf}
\setmonofont{AnonymousPro-Regular.ttf}

\title{Tables, Fonts and Spacing}
\date{}

\begin{document}

\maketitle

\section{Tables}

For simple and short tables there is the handy tabular environment.  The syntax is
\begin{verbatim}
\begin{tabular}{column specification}
  .. & .. & .. \\
  .. & .. & .. \\
\end{tabular}
\end{verbatim}
where \verb~column specification~ is a string containing these characters:
\begin{center}
  \begin{tabular}{cl}
    \verb~l~        & for a column of left aligned text,  \\
    \verb~c~        & for a column of centered text,      \\
    \verb~r~        & for a column of right aligned text, \\
    \verb~|~        & to create a vertical bar, or        \\
    \verb~p{width}~ & for a column with wraparound text of length \verb~width~.
  \end{tabular}
\end{center}
The syntax for moving to the next column and starting a new line is the same syntax as
when using the \verb~\begin{matrix}..\end{matrix}~ command in math mode.  Placing the
optional commands
\begin{verbatim}
\setlength{\tabcolsep}{12pt}
\renewcommand{\arraystretch}{2}
\end{verbatim}
before a table redefines the spacing between columns and rows.  The defaults are
\verb!6pt! and a scaling factor of \verb!1!.

The \verb~\hline~ command creates a horizontal line in the \verb~tabular~ environment.
There are more examples of tables in this document and there are many more examples of
tables of varying complexity on pages 45--48 of the text.

\section{Fonts}

\subsection{Size}

The font size for the entire document is set by the \verb~[11pt]~ option in the
\verb~\documentclass~ command.  This document is typed in 11pt font.  The font size can
also be 10pt (the default option) or 12pt.  Another option in the \verb~\documentclass~
command is \verb~twocolumn~, which has the effect of producing two columns per page.

These commands can be used to change the font size:

\setlength{\tabcolsep}{12pt}
\renewcommand{\arraystretch}{2}

\begin{center}
  \begin{tabular}{ll}
    \verb~\Huge~         & {\Huge Huge}                 \\
    \verb~\huge~         & {\huge huge}                 \\
    \verb~\LARGE~        & {\LARGE LARGE}               \\
    \verb~\Large~        & {\Large Large}               \\
    \verb~\large~        & {\large large}               \\
    \verb~\normalsize~   & {\normalsize normalsize}     \\
    \verb~\small~        & {\small small}               \\
    \verb~\footnotesize~ & {\footnotesize footnotesize} \\
    \verb~\scriptsize~   & {\scriptsize scriptsize}     \\
    \verb~\tiny~         & {\tiny tiny}
  \end{tabular}
\end{center}

The \verb~anyfontsize~ package (not needed if using XeLaTeX) has the
\[\verb~\fontsize{font size}{base line stretch}~\]
command which can be used in conjunction with the \verb~\selectfont~ command to
select size {\fontsize{20mm}{11pt}\selectfont arbitrarily}.

\subsection{Face}

\subsubsection{Selecting the face when using pdfLaTeX}

Font faces are best chosen when compiling with XeLaTeX.  There are a few options for faces
when compiling with pdfLaTeX, some of which carry through to the case of compiling with
XeLaTeX.

Four fonts are needed in any document: Roman (the default font), Math (used for
mathematics), \textsf{Sans serif} (invoked by \verb~\textsf{..}~), and \texttt{Typewriter
  font} (as seen in \verb~verb~ statements).  The standard options for fonts when
compiling with pdfLaTeX can be found when calling these packages:

\setlength{\tabcolsep}{6pt}
\renewcommand{\arraystretch}{1}

\begin{center}
  \begin{tabular}{|lllll|}
    \hline
    Package  & Roman       & Math     & Sans serif  & Typewriter    \\
    \hline
    \hline
    (none)   & CM Roman    & CM Roman & CM Sans     & CM Typewriter \\
    mathpazo & Palatino    & Palatino &             &               \\
    mathptmx & Times       & Times    &             &               \\
    helvet   &             &          & Helvetica   &               \\
    avant    &             &          & Avant Garde &               \\
    courier  &             &          &             & Courier       \\
    chancery & Chancery    &          &             &               \\
    bookman  & Bookman     &          & Avant Garde & Courier       \\
    newcent  & New Century &          & Avant Garde & Courier       \\
    charter  & Charter     &          &             &               \\
    \hline
    fourier  & Utopia      & Fourier  &             &               \\
    euler    &             & Euler    &             &               \\
    \hline
  \end{tabular}
\end{center}

An empty entry indicates that a package does not have an effect on a given font face.  The
last two font selections are listed separately because they are not usually found in the
basic version of \LaTeX{}, but are given by the ``texlive-full'' version.

To select the roman font, math font, sans serif font, and typewriter font separately,
include consecutive \verb~\usepackages~ in a correct order.  For example, the commands
\begin{verbatim}
\usepackage{mathpazo}
\usepackage{charter}
\usepackage{helvet}
\end{verbatim}
uses the Palatino math font, Charter roman font, Helvetica sans serif font, and Computer
Modern Typewriter font.

\subsubsection{Selecting the face when using XeLaTeX}

Any font installed on your computer can be used when compiling with XeLaTeX.  Use commands
similar to these commands before the appearance of \verb~\begin{document}~:

\begin{verbatim}
  \usepackage{mathspec}

  \setmainfont{Lato-Light.ttf}
  \setmathsfont(Digits,Latin,Greek)[Numbers={Lining,Proportional}]{Lato-Light.ttf}
  \setsansfont{351Week4LDFComicSans.ttf}
  \setmonofont{AnonymousPro-Regular.ttf}
\end{verbatim}

These commands change the four required fonts.  To change the mathematics symbols, such
as \(\displaystyle \Sigma, \int, \partial,\) and so on, load the corresponding math package in the
above table before the \verb~mathspec~ package is called.  For example, to use the
symbols that come from \verb~mathpazo~, load \verb~mathpazo~ before \verb~mathspec~.

{\fontsize{4.5ex}{1ex} \fontspec{351Week4LDFComicSans.ttf}Fonts can also be changed
  mid-document.}

\section{Margins and spacing}

\subsection{Margins}

The margins of the document can be controlled with the \verb~geometry~ package.  To set
the left, top, right, and bottom margins to specific values, place a command such as
\begin{verbatim}
\usepackage[left=35mm,top=2cm,right=35mm,bottom=2cm]{geometry}
\end{verbatim}
in the preamble.  The margins used in this document are those values shown above.  As
another example,
\begin{verbatim}
\usepackage[landscape, margin=2in]{geometry}
\end{verbatim}
changes all margins to be 2 inches and prints in landscape mode.

\subsection{Creating Whitespace}

It usually bad form to manually adjust the vertical or horizontal spacing inside the body
of the document when writing an article or book, but it might be appropriate to manually
force white space when creating documents such as syllabi, exams, or resumes.

The line break command \verb~\\~ has an option to add extra space; to add an extra
\verb~1cm~, use \verb~\\[1cm]~.  Alternatively, to force an extra vertical space of
\verb~1cm~ between two paragraphs, the \verb~\vspace{1cm}~ command can be used.  Sometimes
\LaTeX{} will think adding a vertical space using \verb~\vspace~ is a bad idea and won't
cooperate; such a vertical space can be demanded with the command \verb~\vspace*{1cm}~.

The \verb~\vfill~ command produces a length which can stretch or shrink vertically,
pushing the text after the \verb~\vfill~ command as far down the page as possible.  This
command can be used in tandem with the \verb~\newpage~ command, which forces a new page to
begin.  For instance, if writing a mathematics exam, the \LaTeX{} commands
\begin{verbatim}
\begin{problem} Evaluate \( \int \ln x \, dx \). \end{problem}
\vfill
\begin{problem} Evaluate \( \int \sin x \, dx \). \end{problem}
\vfill
\newpage
\end{verbatim}
will produce the next page in the document (provided the \verb~amsthm~ package is loaded
and \verb~\theoremstyle{definition}~ and \verb~\newtheorem{problem}{}~ both appear in the
preamble).  Analogous to \verb~\vspace~, \verb~\vspace*~ and \verb~\vfill~, there are
\verb~\hspace~, \verb~\hspace*~, and \verb~\hfill~ commands, which produce horizontal
space, a forced horizontal space, and a rubber horizontal fill.

\newpage

\begin{problem}
  Evaluate \(\displaystyle \int \ln x \, dx \).
\end{problem}

\vfill

\begin{problem}
  Evaluate \(\displaystyle \int \sin x \, dx \).
\end{problem}

\vfill

\newpage

\subsection{Minipages}

The \verb~minipage~ environment creates a page within a page, useful for side
by side type:

\begin{center}
  \begin{minipage}{15eM}
    This is left text.
  \end{minipage}
  \begin{minipage}{15eM}
    right
  \end{minipage}
\end{center}

\subsection{Unbreakable text}

The compiler makes typesetting decisions by placing boxes around letters, parts of words,
mathematics, and figures, and then appropriately arranging the boxes.  Force unbreakable
type by using \verb~\mbox{unbreakable text}~.  Create a frame around a box using
\verb~\fbox{.}~ and an unbreakable paragraph using \verb~\parbox{width}{.}~.

Sometimes when using commands such as \verb~\hfill~, an empty box is needed to get spacing
just right; create such an empty box with \verb~\mbox{}~.

On a similar note, \verb!~! is a non-breaking space character, used when a space between
two words or characters should appear but those words cannot be on different lines.

\subsection{Tab Stops}

Most of us are familiar with tabs from our frequent use of physical typewriters.  Tabs can
be kept and used with the \verb~tabbing~ environment.  Set tabs using \verb~\=~, create a
new line using \verb~\\~, and move to the next tab using \verb~\>~.  For example,
\begin{tabbing}
  The first tab appears right here,
  \= this is text after the first tab,
  \= and there is a third tab.  \\
  This is the second row,
  \>  middle of second row,
  \> and the end.
\end{tabbing}
Tabbing environments are treated as one box and thus cannot be split across two pages.

\end{document}
