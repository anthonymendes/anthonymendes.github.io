% Created 2023-03-10 Fri 08:35
% Intended LaTeX compiler: xelatex
\documentclass[11pt]{article}



\usepackage{/home/tony/Misc/MendesLaTeXStyle}
\setlist{nolistsep}
\date{}
\title{Markup can generate \LaTeX{}}
\hypersetup{
 pdfauthor={Tony Mendes},
 pdftitle={Markup can generate \LaTeX{}},
 pdfkeywords={},
 pdfsubject={},
 pdfcreator={Emacs 28.2 (Org mode 9.5.5)},
 pdflang={English}}
\begin{document}

\maketitle

\section*{Introduction}
\label{sec:org7aa50bb}

Markup language is a text-encoding system consisting of a set of symbols inserted in a
text document to control its structure, formatting, or the relationship between its parts.

This file is an example of the markup language known as \href{https://en.wikipedia.org/wiki/Org-mode}{org mode}, which comes with the
text editor emacs.  Such documents can easily be converted to other file formats,
including \LaTeX{}.

\subsection*{Including mathematics}
\label{sec:orgafb435c}

Mathematics can be easily inserted, using inline \(\int_a^b x^2 \, dx\) or displayed
equations
\[\int_a^b x^2 \, dx.\]

Unfortunately only simple \LaTeX{} formatting is accepted (no packages, tikz, etc.).

\subsection*{Including lists, tables, and code}
\label{sec:org22de1a5}

Unordered lists can be entered with dashes

\begin{itemize}
\item like
\item this
\item list.
\end{itemize}

Ordered lists can be entered with numbers

\begin{enumerate}
\item like
\item this
\item list.
\end{enumerate}

Code snippets

\lstset{language=LaTeX,label= ,caption= ,captionpos=b,numbers=none}
\begin{lstlisting}
\begin{equation}
  \begin{multline*}
    \alpha + \beta + \gamma + \delta + \varepsilon + \zeta + \eta + \theta   \\
    + \iota + \kappa + \lambda + \mu + \nu + \xi + \pi + \rho \\
    + \sigma + \tau + \upsilon + \varphi + \chi + \psi + \omega.
  \end{multline*}
\end{equation}
\end{lstlisting}

and tables also each have their own syntax.

\begin{center}
\begin{tabular}{llr}
name & date & Score\\
\hline
John & \textit{2023-03-13 Mon} & 3\\
Mary & \textit{2023-03-13 Mon} & 6\\
Keenan & \textit{2023-05-07 Sun} & 1\\
\end{tabular}
\end{center}
\end{document}