% The [10pt] option gives the font size.
\documentclass[10pt]{article}

% '\parindent' controls the indentation at the
% beginning of a paragraph.  The 'eM' is a unit of
% measure in typography that was traditionally derived
% from the width of 'M'.
\setlength{\parindent}{0eM}

% '\parskip' controls the white space between
% paragraphs.  The 'plus' and 'minus' are optional
% specifications that tell the compiler how much
% leeway there is when formatting paragraphs on a
% page.  The 'ex' is a unit of measure equal to the
% height of 'x'.
\setlength{\parskip}{3ex plus 1ex minus 1ex}

\title{\LaTeX{} Typesetting}
\author{\textsc{\textit{mathematics 351 syllabus}}}
\date{}

\begin{document}

\maketitle

% '\thispagestyle{empty}' suppresses the page number
% on this page.  To suppress all page numbers place
% '\pagestyle{empty}' before the '\begin{document}'.
\thispagestyle{empty}

% '\rule{width}{height}' produces a line.
\begin{center}
  \rule{10eM}{1pt}
\end{center}

% The '\@' command that appears after a period tells
% the compiler that the period does not end a sentence
% and should not be spaced as if it did.
\textbf{Instructor:} Anthony Mendes.  Call me Tony or
Dr.\@ Mendes.

\textbf{Email:} \texttt{aamendes@calpoly.edu}

\textbf{Office hours:} Mondays from 12:30pm until 2pm
and Thursdays from 12pm until 1:30pm in Building 25
Room 202.

\textbf{Website}: The website
\texttt{https://anthonymendes.github.io/351.html} has
course information and many \LaTeX{} files.
Assignments are turned in on Canvas.

\textbf{Text}: Our textbook is \textsl{The Not So
  Short Introduction to \LaTeXe} by Oetiker, Partl,
Schlegl, and Hyna.  The book is available on our
website.  Some course topics cannot be found in the
text.

\textbf{Content:} Each week we will introduce new
\LaTeX{} concepts.  By the end of the course, you will
be able to use \LaTeX{} to create professional looking
research articles, presentations, posters, classroom
materials, and r\'esum\'es.

\textbf{Assignments:} Each week there will be an
assignment.  All source files and the \texttt{.pdf}
output for each assignment are to be submitted to
Canvas.

\textbf{Grading:} A student who completes 7 out of the
9 assignments on time and with acceptable quality will
be assigned ``Credit'' for the course.  Otherwise a
student will receive ``No Credit''.  There are no
exams.

\textbf{\LaTeX{er} of the week:} Each week one student
with an exemplary assignment will be recognized as our
``\LaTeX{er} of the week''.  Their name and assignment
will be located in a place of honor on our web site
for all to envy and admire.  Please let me know if you
do not want your name or work on our web site.

\rule{\textwidth}{.02ex}

\end{document}