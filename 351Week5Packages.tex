\documentclass[11pt]{paper}

\usepackage{351Week5Package}

\title{Packages and Classes}
\date{}

\begin{document}

\maketitle

\section{Newcommands}

Almost everything in \LaTeX{} can be customized, including \LaTeX{} commands
themselves.  To create a new command, place
\begin{lstlisting}
\newcommand{\name}{definition}
\end{lstlisting}
in the preamble.  Then, to call the command, use \verb~\name~ in the document.
The compiler will complain if \verb~\name~ is a predefined \LaTeX{} command.
The \verb~\newcommand~ command has an optional parameter to include input.  For
instance,
\begin{lstlisting}
\newcommand{\integral}[2]{$\int_{\mathbb{R}} #1 \, d#2$}
\end{lstlisting}
defines the command \verb~\integral~ which takes in 2 inputs, placed where the
\verb~#1~ and \verb~#2~ appear in the definition.  Calling
\verb~\integral{\sin x}{x}~ now produces \integral{\sin x}{x}.

For function names in math mode which are not predefined (such as $\ln x$,
$\sin x$, $\arctan x$), use a command such as
\verb~\DeclareMathOperator{\dimension}{dim}~ to define a command
\verb~\dimension~.  This produces ``$\dimension$'', a math mode symbol.

To redefine a previously defined \LaTeX{} command, use the syntax
\begin{verbatim}
\renewcommand{\old}{\new}
\end{verbatim}
For instance, \verb~\renewcommand{\phi}{\varphi}~ changes the appearance of
$\phi$ throughout the document.  As another example, this command can be used to
change the end-of-proof symbol in the \verb~amsthm~ proof environment into
creating a black square pushed to the right of the line by placing this into
the preamble:
\begin{lstlisting}
\renewcommand{\qed}{\hfill \( \blacksquare \)}
\end{lstlisting}

Analogous to \verb~\newcommand~, there is a \verb~\newenvironment~ command to
create custom environments.  The syntax to be placed in the preamble is
\begin{lstlisting}
\newenvironment{name}{before}{after}
\end{lstlisting}
To call the command, use \verb~\begin{name} .. \end{name}~.  Then, commands in
\verb~before~ are run before \verb~..~ and commands in \verb~after~ are run
after \verb~..~.  Just like \verb~\newcommand~, there is an option for up to
$9$ input variables.

Especially when reusing the same preamble for multiple documents, it may be
convenient to store the preamble in a file with the \verb~.sty~ extension (the
\verb~.sty~ stands for ``style file'').  In the first line of the style file,
place the command
\begin{lstlisting}
\ProvidesPackage{351Week5Package}
\end{lstlisting}
and then call the package just like any other package with the
\verb~\usepackage~ command in the preamble of the main document.

\section{Packages}

Packages can also be used to extend the functionality of \LaTeX{} in some way.
The packages we have introduced in our course so far, listed below, are among
the most frequently used \LaTeX{} packages.

\begin{center}
  \begin{tabular}{ll}
    \toprule
    Package          & Purpose                            \\
    \midrule
    \ctan{amsmath}   & Typesetting mathematics            \\
    \ctan{amssymb}   & Math symbols and fonts             \\
    \ctan{amsthm}    & Theorem and proof environments     \\
    \ctan{geometry}  & Control page margins               \\
    \ctan{makeindex} & Create and index                   \\
    \ctan{mathspec}  & For including fonts (with XeLaTeX) \\
    \bottomrule
  \end{tabular}
\end{center}

There are over 5000 more \LaTeX{} packages!  Other widely used packages include:

\begin{center}
  \begin{tabular}{l l}
    \toprule
    Package              & Purpose                                            \\
    \midrule
    \ctan{hyperref}      & Hyperlinks and clickable references                \\
    \ctan{enumitem}      & Improved enumerate and itemize environment         \\
    \ctan{booktabs}      & Improved tabular environment                       \\
    \ctan{IEEEtrantools} & Improved multiline math (see page 64 of the text)  \\
    \ctan{fancyhdr}      & Improved headers and footers                       \\
    \ctan{babel}         & Support for other languages                        \\
    \ctan{listings}      & For typesetting computer code                      \\
    \ctan{graphicx}      & To include outside graphics                        \\
    \ctan{pgf}           & To create graphics (TikZ)                          \\
    \ctan{natbib}        & An alternative to BibTeX                           \\
    \ctan{microtype}     & Micro-typographic extensions (only with pdfLaTeX)  \\
    \ctan{textpos}       & Absolute positioning of text                       \\
    \ctan{pgfornament}   & Ornamental flourishes                              \\
    \bottomrule
  \end{tabular}
\end{center}

Most of the packages listed here are shipped with many versions of \LaTeX{} and
probably can be accessed using \verb~\usepackage{name}~ in the preamble.  If
they are not already installed, these and many other packages can be downloaded
from the ``Comprehensive \TeX{} Archive Network'', online at
\url{https://www.ctan.org}.  This is also where you can find the documentation
for the packages listed above.  \textbf{The first step in using any of the
  above packages is to actually read the documentation!}

How to use a package that is not already installed:
% These are options allowed by the enumitem package
\begin{enumerate}[labelindent = \parindent, leftmargin = *, label={\arabic*.}]
\item Find the package at \url{https://www.ctan.org} or elsewhere.
\item \textbf{Read the readme or the documentation.}
\item See pages 89--90 of the text on how to install.  Another good resource on
  how to install an extra package is at
  \url{https://en.wikibooks.org/wiki/LaTeX/Installing_Extra_Packages}.  As a
  shortcut, if all that is needed is a \verb~.sty~ file, then you can try
  copying the \verb~.sty~ files into the folder which contains the \verb~.tex~
  file.
\item Use by including \verb~\usepackage{name}~ in the preamble.
\end{enumerate}
Some third party software packages automate this procedure, possibly doing it
automatically as soon as a package is loaded with \verb~\usepackage~ in the
preamble.

It is considered bad form to load many packages and then not use them.  Loading
obscure packages makes the \verb~.tex~ less portable and increases the chance
that packages will conflict with one another.  Packages also tend to become
obsolete.  As a general rule, use a minimum number of packages.

\section{Classes}

Class files are loaded by placing the \verb~\documentclass{class}~ command in
the first line of the \verb~.tex~ file.  Classes tend to have their own
specialized commands; for example, the familiar \verb~article~ class provides
commands such as \verb~\section~, \verb~\tableofcontents~, and \verb~\author~.

Although so far we have only used the \ctan{article} class, there are many
other class files.  Some of the more popular packages are:
\begin{center}
  \begin{tabular}{l l}
    \toprule
    Class             & Purpose                                                      \\
    \midrule
    \ctan{amsart}     & \texttt{article} alternative                                 \\
    \ctan{paper}      & \texttt{article} alternative (used in this document)         \\
    \ctan{book}       & Books                                                        \\
    \ctan{memoir}     & \texttt{book} alternative; a great choice for books/theses   \\
    \ctan{letter}     & Formal letters                                               \\
    \ctan{scrlttr2}   & \texttt{letter} alternative; one of many Koma-Script classes \\
    \ctan{moderncv}   & Curriculum vitae                                             \\
    \ctan{beamer}     & Presentation slides                                          \\
    \ctan{tikzposter} & Conference posters                                           \\
    \ctan{exam}       & Exams                                                        \\
    \ctan{standalone} & cropped \texttt{.pdf} output (good for TikZ)                 \\
    \bottomrule
  \end{tabular}
\end{center}

If not already present on your system, class files can be found on
\url{https://www.ctan.org} and installed in a similar way that packages are
installed.  Sometimes it is possible to simply place the desired \verb~.cls~
file in the folder containing the \verb~.tex~ file.

Of course one should \textbf{read the documentation} and look at example files
to learn how to use any particular class!

\end{document}
