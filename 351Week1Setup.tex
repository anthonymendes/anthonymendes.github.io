\documentclass[11pt]{article}

\title{How to compile \LaTeX{} files}
\date{}

\begin{document}

% Comments are written like this.
\maketitle

\LaTeX{} files can be edited and compiled locally on
your personal computer.  This is the preferred option
for serious projects such as books, research
publications and presentations.  There are many free
options for downloading and installing \LaTeX{} on
each operating system.
\begin{itemize}
\item Windows users can use MikTeX.  See
  \texttt{miktex.org/}.

\item Apple users can use MacTex.  See
  \texttt{www.tug.org/mactex/}.

\item Linux users may find \LaTeX{} is already
  installed on their machine.  If not, Debian or
  Ubuntu users can enter \texttt{sudo apt-get install
    texlive-full} in a terminal.
\end{itemize}
\LaTeX{} files can also be edited and compiled on a
third party web site such as
\texttt{www.overleaf.com}.  This is a fine option to
get up and running quickly.

Use \LaTeX{} by first opening and editing a plain text
file.  Save the file with the \texttt{.tex} extension.
The \texttt{.tex} file can be compiled to produce a
\texttt{.pdf} file using either \texttt{pdflatex} or
\texttt{xelatex}.  The choice of compiler is usually
an option in a drop down menu in \LaTeX{} software
packages or, if using the command line, can be invoked
by entering \texttt{pdflatex filename.tex} or
\texttt{xelatex filename.tex} in a terminal.  The
\texttt{pdflatex} option is faster and more
established, but the \texttt{xelatex} option is better
at handling fonts.

I use Emacs with the Auctex package to edit
\texttt{.tex} files and compile with \texttt{xelatex}.
This is probably the best way to write \LaTeX{} code,
but requires knowledge of Emacs and is not for
beginners.

\end{document}
