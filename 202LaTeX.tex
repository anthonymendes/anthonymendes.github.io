\documentclass[11pt]{article}

\usepackage{amssymb,amsmath,amsthm}

\title{\LaTeX{} basics}
\date{}

\begin{document}

% Comments are written like this.
\maketitle

\section{Creating documents}

Text!!! $\epsilon$
$\sqrt{x^2+1}$
\( \tau \int  \sum_{ n = 1 }^{ \infty} \)
\begin{equation*}
\sum_{ n = 1 }^{ \infty}
\end{equation*}

Use \LaTeX{} by first opening and editing a plain text file.  Save the file with the
\texttt{.tex} extension.  The \texttt{.tex} file can be compiled to produce a
\texttt{.pdf} file using either \texttt{pdflatex} or \texttt{xelatex}.  The choice of
compiler is usually an option in a drop down menu in \LaTeX{} software packages or, if
using the command line, can be invoked by entering \texttt{pdflatex filename.tex} or
\texttt{xelatex filename.tex} in a terminal.  The \texttt{pdflatex} option is faster and
more established, but the \texttt{xelatex} option is better at handling fonts.

\LaTeX{} files can also be edited and compiled on a third party web site such as
\texttt{www.overleaf.com}.  This is a fine option to get up and running quickly, but most
serious users install LaTeX on their personal computer.  There are many free options for
downloading and installing \LaTeX{} on each operating system:
\begin{itemize}
\item Windows users can use MikTeX.  See
  \texttt{miktex.org/}.

\item Apple users can use MacTex.  See
  \texttt{www.tug.org/mactex/}.

\item Linux should be able to figure it out themselves.
\end{itemize}

\section{Entering mathematics}

Mathematics can be typeset using the packages \texttt{amssymb},
\texttt{amsmath} and \texttt{amsthm}.  These packages are maintained by the
American Mathematical Society.

To produce inline ``textstyle'' mathematics, place mathematics between the
symbols \verb`\(` and \verb`\)` or between the symbols \verb`$` and \verb`$`.
To produce ``displaystyle'' mathematics on its own centered line, similar to
\begin{equation*}
  \frac{d}{dx} \arcsin x = \frac{1}{\sqrt{1 - x^2}},
\end{equation*}
place mathematics between \verb`\[` and \verb`\]` or between
\verb`\begin{equation*}` and \verb`\end{equation*}`.  This is known as
``displaystyle''.  Inline mathematics can appear as displaystyle using the
\verb!{\displaystyle .. }! command and displayed mathematics can appear as
textstyle using \verb`{\textstyle .. }`.

Mathematics symbols are typeset using commands such as \verb~\(\Xi\)~,
producing \(\Xi\).  \LaTeX{} will complain if such a mathematics symbol is not
called in math mode.  A list of mathematics symbols can be found on pages 75--82
of our text.  The web site
\begin{center}
  \texttt{http://detexify.kirelabs.org/classify.html}
\end{center}
can also help in finding symbols.  \emph{Use standard AMS packages whenever
  possible!}

The most frequently encountered functions and operators in mathematics have
pre-defined command names.  For example:
\[
  \sin x, \cos x, \tan x, \arcsin x, \ln x
\]
Use \verb~\operatorname{..}~ if a function or operator does not have a
predefined command, like this: \(\operatorname{erf}x\).

Binomial coefficients:
\({\displaystyle 1 + 2 + \cdots + n = \binom{n+1}{2}}\), and matrices:
\begin{equation*}
  \begin{bmatrix}
    a & \dots & b \\
    \vdots  & \ddots & \\
    -3 & \dots & 4     \\
  \end{bmatrix}
\end{equation*}
\begin{equation*}
  \begin{bmatrix}
    a & b \\
    c & d \\
  \end{bmatrix}
  \qquad
  \begin{vmatrix}
    a & b \\
    c & d
  \end{vmatrix}
  \qquad
  \begin{pmatrix}
    a & b \\
    c & d
  \end{pmatrix}
  \qquad
  \begin{matrix}
    a & b \\
    c & d
  \end{matrix}
  \qquad
  \begin{smallmatrix}
    a & b \\
    c & d \\
  \end{smallmatrix}
  \qquad
  \left[
  \begin{smallmatrix}
    a & b \\
    c & d \\
  \end{smallmatrix}
  \right]
  \qquad
  \begin{Vmatrix}
    a & b \\
    c & d
  \end{Vmatrix}
\end{equation*}
By convention, matrices are not written in boldface but vectors such as
\(\mathbf{x}\) are.  To typeset a transpose, use the \verb~\intercal~ symbol, such as
\(\mathbf{x}^\intercal\).

\section{Spacing}

Sometimes the spacing between mathematics symbols should be adjusted.  The
spacing commands to be used in math mode, in order of longest space to smallest
space, are:
\[
  \verb~\qquad~ \qquad
  \verb~\quad~  \qquad
  \verb~\~      \qquad
  \verb~\;~     \qquad
  \verb~\:~     \qquad
  \verb~\,~     \qquad
  \verb~\!~     \qquad
\]
These produce spaces equal to the width of two eMs, one eM (one quad), the
width of inter-word spacing, \(\frac{5}{18}\) quad, \(\frac{4}{18}\) quad,
\(\frac{3}{18}\) quad, and \(-\frac{3}{18}\) quad.  It is proper form to use
\verb~\,~ before differentials in integrals, like this
\[
  \int_{-\infty}^\infty \int_{-\infty}^\infty e^{-(x^2+y^2)} \, dx \, dy
  = \int_0^{2 \pi} \int_0^\infty e^{-r^2} r \, dr \, d\theta.
\]
Here is the incorrect typesetting, without the proper spacing:
\[
  \int_{-\infty}^\infty \int_{-\infty}^\infty e^{-(x^2+y^2)}  dx  dy
  = \int_0^{2 \pi} \int_0^\infty e^{-r^2} r dr d\theta.
\]

In rare cases spacing can be adjusted using \verb~\phantom{stuff}~ which
creates a space of the same length as the typeset length of \verb~stuff~.
Using \verb~\phantom~ frequently is a sign of poor typesetting.

Parentheses of the correct size are given by \verb~\left( .. \right)~.  The
parentheses do not have to be the same type.  To suppress a parentheses,
replace the parenthesis with a period.  For example,
\[ \left \{ x \in \mathbb{R} : \int_0^x \sin t \, dt \leq 1 \right \}. \]

\section{Multi-line mathematics}

To display multi-line math example, use \verb~\begin{align*}..\end{align*}~,
where the \verb~&~ symbol controls where the alignment occurs and \verb~\\~
gives a new line.  For instance, consider
\begin{align}
  \int_a^b x^n \, dx
 & = \left. \frac{x^{n+1}}{n+1} \right|_a^b   \\
 & = \frac{b^{n+1}}{n+1} -\frac{a^{n+1}}{n+1} \nonumber \\
 & = \frac{1}{n+1}(b^{n+1}-a^{n+1}).
\end{align}

For long expressions that don't fit on one line, use
\verb~\begin{multline*}~, with \verb~\\~ denoting a new line:

\begin{multline*}
  \alpha + \beta + \gamma + \delta + \varepsilon + \zeta + \eta + \theta   \\
  + \iota + \kappa + \lambda + \mu + \nu + \xi + \pi + \rho \\
  + \sigma + \tau + \upsilon + \varphi + \chi + \psi + \omega.
\end{multline*}

Fractions can be displayed a couple of ways: \(\frac{1}{2}\) versus \(\dfrac{1}{2}\).

\end{document}
