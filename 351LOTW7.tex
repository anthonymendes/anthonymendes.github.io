\documentclass[landscape]{tikzposter}
% tikzposter automatically loads tikz, graphicx

% Themes: Rays, Basic, Simple, Envelope, Wave, Board, Autumn, Desert
\usetheme{Desert}

% Color styles: Default, Australia, Britain, Sweden, Spain, Russia, Denmark, Germany
\usecolorstyle{Spain}

% Background styles: Default, Rays, VerticalGradation, BottomVerticalGradation, Empty
\usebackgroundstyle{VerticalGradation}

% Title styles: Default, Basic, Envelope, Wave, VerticalShading, Filled, Empty
\usetitlestyle{Wave}

% Block styles: Default, Basic, Minimal, Envelope, Corner, Slide, TornOut
\useblockstyle{Slide}

% To removes logo in bottom right corner
\tikzposterlatexaffectionproofoff

% See the tikzposter manual for more custom colors and styles

\usepackage{amsmath,amssymb, mathspec}

\usepackage[pdftitle = {Assignment 7},
pdfauthor = {Joshua Smith},
pdfsubject = {Miscellaneous},
colorlinks = true,
urlcolor = blue,
linkcolor = blue,
citecolor = blue]{hyperref}

% \textup disables the default small caps in title
% Change font to a font on your system or do not select font
\title{\fontspec{TeX Gyre Bonum} \selectfont
  \fontsize{1.5in}{1ex} \textup{\textbf{Pedro Pascal}}}
\author{Joshua Smith}
\institute{Cal Poly}

\begin{document}

\maketitle

\begin{columns}
  \column{0.66}
  \block{Very Interesting Trivia}{
    \vspace{1ex}
    \begin{itemize}
    \item Pedro was a \textbf{competitive swimmer} as a child, even competing in the Texas state championships as an 11-year-old! \\[1ex]
    \item I've seen him in \\[1ex]
        \textit{The Mandalorian} (2019 -- 2023) \\
        \textit{Wonder Woman 1984} (2020) \\
        \textit{The Book of Boba Fett} (2022) \\
        \textit{Triple Frontier} (2019) \\
        ... \\
        \textit{and more!} \\[1ex]
    \item Pedro uses his \textbf{mother's name professionally} rather than his father's to honor her impact on his life. \\[1ex]
    \item Pedro Pascal and fellow actor Oscar Isaac met \textbf{way back in 2005} working on an off-broadway production \\ making just a couple hundred of dollars each week, and they are \textbf{still friends today}! \\[2ex]
    \end{itemize}}
  \note[width = 13cm, angle = 10, radius = 30cm, rotate = -15]{
    A great video on YouTube from LADbible showed me lots of Pedro's favorite Chilean snacks! 10/10 excellent watch.
    \begin{center}
        \includegraphics[height = 3in]{351LOTW7a.jpg}
    \end{center}
    }

  \column{.34}
  \block{My Pedro Watch-list}{
    \begin{itemize}
    \item \textit{The Unbearable Weight of Massive Talent}
    (2022, with Nick Cage!) \\[1ex]
    \item \textit{Narcos} (2015 --- I loved \textit{Breaking Bad}, I think I will enjoy this also) \\[1ex]
    \item \textit{The Last of Us} (2023). I normally would never purchase HBO Max, but the clips I see online have got me seriously considering dropping \$150 annually\ldots
    \end{itemize}}
\end{columns}

\begin{columns}
  \column{0.33}
  \block{Great Social Media Presence}
  {While he tragically recently deactivated his twitter account, \\
  Pedro's internet thought wall was a constant supply of fun. \\[2ex]

        \includegraphics[height = 5in]{351LOTW7b.jpg}
    }

  \column{0.67}
  \block{My Favorite Graphs}{
    \begin{center}
      It may seem off topic, but secretly I know Pedro loves graphing different functions as much as me\ldots \\[2ex]
      \begin{tikzpicture}[scale = 2.5]
        \node at (0,-3) {A classic parabola};
        \draw [->, thin] (-1.5,0) -- (2.5,0);
        \draw [->, thin] (0,-2) -- (0,2);
        \draw (1, .05) -- (1, -.05) node [below] {$1$};
        \draw [domain=-1:2, samples = 100, smooth, ultra thick, blue]
        plot (\x, {-1*pow(\x, 2) + \x + 1});
      \end{tikzpicture}
      \hspace{5ex}
      \begin{tikzpicture}[scale = 2.5]
        \node at (0,-3) {A simple sine wave};
        \draw [->, thin] (-1.5,0) -- (6.5,0);
        \draw [->, thin] (0,-2) -- (0,2);
        \draw (1, .05) -- (1, -.05) node [below] {$1$};
        \draw [domain=-1:2*pi, samples = 100, smooth, ultra thick, blue]
        plot (\x, {sin(\x r)});
      \end{tikzpicture}
      \hspace{5ex}
      \begin{tikzpicture}[scale = 2.5]
        \node at (0,-3) {Super silly polynomial!};
        \draw [->, thin] (-1.5,0) -- (2.5,0);
        \draw [->, thin] (0,-2) -- (0,2);
        \draw (1, .05) -- (1, -.05) node [below] {$1$};
        \draw [domain=-1.1:2.1, samples = 100, smooth, ultra thick, blue]
        plot (\x, {(\x - 1) * (\x) * (\x + 1) * (\x - 2)});
      \end{tikzpicture}
      \hspace{5ex}
      \begin{tikzpicture}[scale = 2.5]
        \node at (0,-3) {Not a function, but circles are cool!};
        \draw [->, thin] (-1.5,0) -- (1.5,0);
        \draw [->, thin] (0,-1.5) -- (0,1.5);
        \draw (1, .05) -- (1, -.05) node [below] {$1$};
        \draw [domain=0:2*pi, samples = 100, smooth, ultra thick, blue]
        plot ({cos(\x r)},
        {sin(\x r)});
      \end{tikzpicture}
    \end{center}}
\end{columns}

\end{document}
