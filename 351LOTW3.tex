\documentclass[10pt]{article}

\usepackage{amssymb,amsmath,amsthm,makeidx}

\makeindex

\newtheorem{theorem}{Theorem}
\newtheorem{lemma}[theorem]{Lemma}

\title{\LaTeX{} Assignment 3}
\author{Samuel Sehnert}
\date{2023/02/02}

\setlength{\parindent}{0eM}
\setlength{\parskip}{1ex}

\begin{document}
    \maketitle
    \tableofcontents
    \section{Instructions}
        There are two exercises (which were typeset using the theorem environment).

        \textbf{Exercise 1.} \textit{Recreate this entire document.\footnote{How Meta}}

        \textbf{Exercise 2.} \textit{Create a new document containing a short description of three
            of your favorite books, papers, or other publications. Be sure to include a
            bibliography, created using BibTeX.}

        \setlength{\parindent}{2eM}
        An assignment which completes Exercise 2 in an interesting way or makes
        amusing use of mathematical typesetting will earn the coveted \LaTeX{}er of the
        week distinction
    \subsection[Due Date]{When to turn it in}
        Please upload the \verb|.tex| and \verb|.bib| source files and the \verb|.pdf| output files to your
        solutions to Assignment 3 on or before Sunday, January 26.
    \section{Euler was smart}
        Euler proved many statements, such as
        \begin{equation}
            \label{Euler_Pent_Num}
            \prod_{m=1}^{\infty}\left(1-q^m\right)=\sum_{n=-\infty}^{\infty}\left(-1\right)^n q^{\left(2n^2-n\right)/2},
        \end{equation}
        where q is an indeterminate. Equation \eqref{Euler_Pent_Num} is known as Euler's pentagonal number theorem.
        Euler also proved Theorem 1 below.
        \begin{theorem}[The Basel Problem]
            we have $\sum_{n=1}^{\infty}\frac{1}{n^2}=\frac{\pi^2}{6}.$
        \end{theorem}
        Euler’s original proof of Theorem 1 makes unjustified assumptions that infinite
        products and sums behave like finite products and sums, but is interesting
        nonetheless and worth displaying.
        \begin{proof}
            Using the power series for $\sin{x}$, we have
            \begin{align*}
                \frac{\sin x}{x} & = \frac{1}{x}\left(x - \frac{x^3}{3!}+\frac{x^5}{5!}-\ldots\right) \\
                                 & = 1 - \frac{x^2}{3!}+\frac{x^4}{5!}-\ldots \\
                                 \label{proof_0}
                                 \tag{2}
                                 & = \left(1-\frac{x}{\pi}\right) \left(1+\frac{x}{\pi}\right) \left(1-\frac{x}{2\pi}\right)\left(1+\frac{x}{2\pi}\right)\ldots
            \end{align*}
            where te reasoning \footnote{This reasoning is actually true, but needs further justification.}
            behind \eqref{proof_0} is that a polynomial can be factored if its roots
            are known, and the roots of $\sin x/x$ are $\pm\pi, \pm2\pi$,\ldots. Multiplying each pair
            of consecutive terms in this product gives
            \begin{equation*}
                \label{proof_1}
                \tag{3}
                \left(1-\frac{x^2}{\pi^2}\right) \left(1-\frac{x^2}{4\pi^2}\right) \left(1-\frac{x^2}{9\pi^2}\right)\ldots
            \end{equation*}
            The coefficient of $x^2$ in \eqref{proof_1} is $\displaystyle -\frac{1}{\pi^2}-\frac{1}{4\pi^2}-\ldots=-\frac{1}{\pi^2}\sum_{n=1}^{\infty}\frac{1}{n^2}$ and the
            coefficient of $x^2$ is $\sin x/x$ is $-1/3!$, so equating these two expressions proves the theoem.
        \end{proof}
\end{document}
